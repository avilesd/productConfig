\documentclass[]{article}
\usepackage{lmodern}
\usepackage{amssymb,amsmath}
\usepackage{ifxetex,ifluatex}
\usepackage{fixltx2e} % provides \textsubscript
\ifnum 0\ifxetex 1\fi\ifluatex 1\fi=0 % if pdftex
  \usepackage[T1]{fontenc}
  \usepackage[utf8]{inputenc}
\else % if luatex or xelatex
  \ifxetex
    \usepackage{mathspec}
    \usepackage{xltxtra,xunicode}
  \else
    \usepackage{fontspec}
  \fi
  \defaultfontfeatures{Mapping=tex-text,Scale=MatchLowercase}
  \newcommand{\euro}{€}
\fi
% use upquote if available, for straight quotes in verbatim environments
\IfFileExists{upquote.sty}{\usepackage{upquote}}{}
% use microtype if available
\IfFileExists{microtype.sty}{%
\usepackage{microtype}
\UseMicrotypeSet[protrusion]{basicmath} % disable protrusion for tt fonts
}{}
\usepackage[margin=1in]{geometry}
\usepackage{color}
\usepackage{fancyvrb}
\newcommand{\VerbBar}{|}
\newcommand{\VERB}{\Verb[commandchars=\\\{\}]}
\DefineVerbatimEnvironment{Highlighting}{Verbatim}{commandchars=\\\{\}}
% Add ',fontsize=\small' for more characters per line
\usepackage{framed}
\definecolor{shadecolor}{RGB}{248,248,248}
\newenvironment{Shaded}{\begin{snugshade}}{\end{snugshade}}
\newcommand{\KeywordTok}[1]{\textcolor[rgb]{0.13,0.29,0.53}{\textbf{{#1}}}}
\newcommand{\DataTypeTok}[1]{\textcolor[rgb]{0.13,0.29,0.53}{{#1}}}
\newcommand{\DecValTok}[1]{\textcolor[rgb]{0.00,0.00,0.81}{{#1}}}
\newcommand{\BaseNTok}[1]{\textcolor[rgb]{0.00,0.00,0.81}{{#1}}}
\newcommand{\FloatTok}[1]{\textcolor[rgb]{0.00,0.00,0.81}{{#1}}}
\newcommand{\CharTok}[1]{\textcolor[rgb]{0.31,0.60,0.02}{{#1}}}
\newcommand{\StringTok}[1]{\textcolor[rgb]{0.31,0.60,0.02}{{#1}}}
\newcommand{\CommentTok}[1]{\textcolor[rgb]{0.56,0.35,0.01}{\textit{{#1}}}}
\newcommand{\OtherTok}[1]{\textcolor[rgb]{0.56,0.35,0.01}{{#1}}}
\newcommand{\AlertTok}[1]{\textcolor[rgb]{0.94,0.16,0.16}{{#1}}}
\newcommand{\FunctionTok}[1]{\textcolor[rgb]{0.00,0.00,0.00}{{#1}}}
\newcommand{\RegionMarkerTok}[1]{{#1}}
\newcommand{\ErrorTok}[1]{\textbf{{#1}}}
\newcommand{\NormalTok}[1]{{#1}}
\usepackage{graphicx}
\makeatletter
\def\maxwidth{\ifdim\Gin@nat@width>\linewidth\linewidth\else\Gin@nat@width\fi}
\def\maxheight{\ifdim\Gin@nat@height>\textheight\textheight\else\Gin@nat@height\fi}
\makeatother
% Scale images if necessary, so that they will not overflow the page
% margins by default, and it is still possible to overwrite the defaults
% using explicit options in \includegraphics[width, height, ...]{}
\setkeys{Gin}{width=\maxwidth,height=\maxheight,keepaspectratio}
\ifxetex
  \usepackage[setpagesize=false, % page size defined by xetex
              unicode=false, % unicode breaks when used with xetex
              xetex]{hyperref}
\else
  \usepackage[unicode=true]{hyperref}
\fi
\hypersetup{breaklinks=true,
            bookmarks=true,
            pdfauthor={Diego Aviles},
            pdftitle={3.3 Illustration of the productConfig package},
            colorlinks=true,
            citecolor=blue,
            urlcolor=blue,
            linkcolor=magenta,
            pdfborder={0 0 0}}
\urlstyle{same}  % don't use monospace font for urls
\setlength{\parindent}{0pt}
\setlength{\parskip}{6pt plus 2pt minus 1pt}
\setlength{\emergencystretch}{3em}  % prevent overfull lines
\setcounter{secnumdepth}{0}

%%% Use protect on footnotes to avoid problems with footnotes in titles
\let\rmarkdownfootnote\footnote%
\def\footnote{\protect\rmarkdownfootnote}

%%% Change title format to be more compact
\usepackage{titling}

% Create subtitle command for use in maketitle
\newcommand{\subtitle}[1]{
  \posttitle{
    \begin{center}\large#1\end{center}
    }
}

\setlength{\droptitle}{-2em}
  \title{3.3 Illustration of the productConfig package}
  \pretitle{\vspace{\droptitle}\centering\huge}
  \posttitle{\par}
  \author{Diego Aviles}
  \preauthor{\centering\large\emph}
  \postauthor{\par}
  \predate{\centering\large\emph}
  \postdate{\par}
  \date{2016-03-28}



\begin{document}

\maketitle


\begin{center}\rule{0.5\linewidth}{\linethickness}\end{center}

First, let us look at the data:

\begin{Shaded}
\begin{Highlighting}[]
\KeywordTok{tail.matrix}\NormalTok{(camera_data)}
\end{Highlighting}
\end{Shaded}

\begin{verbatim}
##       cid usid round atid   selected selectable
## 1823 1835   62     1    4 0.01944444          1
## 1824 1836   62     2    4 0.16805556          1
## 1825 1837   63     0    1 1.00000000          2
## 1826 1838   63     0    2 1.00000000          1
## 1827 1839   63     0    3 1.00000000          1
## 1828 1840   63     0    4 0.16805556          1
\end{verbatim}

As you can see our data displays 1828 rows with around 63 different
users in a rather complex format which makes it practically difficult to
work with. This is the reason we need the basic function cluster
\texttt{GetFunctions}. For example, it is quite necessary to know how
many attributes there are in out data:

\begin{Shaded}
\begin{Highlighting}[]
\KeywordTok{get_attrs_ID}\NormalTok{(}\DataTypeTok{dataset=}\NormalTok{camera_data)}
\end{Highlighting}
\end{Shaded}

\begin{verbatim}
## [1] 1 2 3 4
\end{verbatim}

Given that our functions are mostly vectorized and assuming all users
have the same attribtues, we can ask for the unique values of each
\texttt{attr}.

\begin{Shaded}
\begin{Highlighting}[]
\KeywordTok{getAttrValues}\NormalTok{(}\DataTypeTok{dataset=}\NormalTok{camera_data, }\DataTypeTok{attr =} \KeywordTok{c}\NormalTok{(}\DecValTok{1}\NormalTok{,}\DecValTok{2}\NormalTok{,}\DecValTok{3}\NormalTok{,}\DecValTok{4}\NormalTok{))}
\end{Highlighting}
\end{Shaded}

\begin{Shaded}
\begin{Highlighting}[]
\KeywordTok{lapply}\NormalTok{(temp, unique)}
\end{Highlighting}
\end{Shaded}

\begin{verbatim}
## $`1`
## [1] 3 0 2 1
## 
## $`2`
## [1] 0 3 2 1
## 
## $`3`
## [1] 0 3 2 1
## 
## $`4`
## [1]  0.16805556 -0.27777778 -0.12916667  0.01944444  0.46527778  0.31666667
## [7]  0.61388889
\end{verbatim}

Now that we know how many attributes there are, we also know how many
columns the decision matrices are going to have. The number of rows
depends on how much each user interacted with the product configurator
and again, since functions are vectorised we can calculate the number of
rows for all users using \texttt{getRoundsById}.

\begin{Shaded}
\begin{Highlighting}[]
\NormalTok{all.rounds <-}\StringTok{ }\KeywordTok{getRoundsById}\NormalTok{(camera_data, }\DataTypeTok{userid =} \KeywordTok{getAllUserIds}\NormalTok{(camera_data))}
\KeywordTok{head}\NormalTok{(all.rounds, }\DecValTok{3}\NormalTok{) }\CommentTok{# To display only the results for the first three users}
\end{Highlighting}
\end{Shaded}

\begin{verbatim}
## $`6`
##   [1]  0  1  2  3  4  5  6  7  8  9 10 11 12 13 14 15 16 17 18 19 20 21 22
##  [24] 23 24 25 26 27 28 29 30 31 32 33 34 35 36 37 38 39 40 41 42 43 44 45
##  [47] 46 47 48 49 50 51 52 53 54 55 56 57 58 59 60 61 62 63 64 65 66 67 68
##  [70] 69 70 71 72 73 74 75 76 77 78 79 80 81 82 83 84 85 86 87 88 89 90 91
##  [93] 92 93 94 95 96 97 98 99
## 
## $`9`
## [1] 0 1 2 3
## 
## $`10`
## [1] 0 1 2 3 4
\end{verbatim}

We can now easily observe that user 10 interacted four times with the
configurator four times before making a decision.

The three functions presented above are necessary to create more complex
structures, such as the decision matrix. To build it, we just need to
use the right function with the right parameters. At mentioned earlier,
the fourth parameter \texttt{attr=4} is price, which means it is a cost
attribute (lower values are better). To handle this we input the
correspondent attribute ID in \texttt{cost\_ids}. Choosing any random
user from our table, we calculate its decision matrix.

\begin{Shaded}
\begin{Highlighting}[]
\KeywordTok{decisionMatrix}\NormalTok{(camera_data, }\DecValTok{33}\NormalTok{, }\DataTypeTok{rounds=}\StringTok{"all"}\NormalTok{, }\DataTypeTok{cost_ids=}\DecValTok{4}\NormalTok{)}
\end{Highlighting}
\end{Shaded}

\begin{verbatim}
## $`33`
##        attr1 attr2 attr3       attr4
## 0round     1     1     1 -0.16805556
## 1round     1     1     2 -0.01944444
## 2round     1     0     2 -0.16805556
## 3round     2     0     2 -0.01944444
## 4round     1     0     2 -0.16805556
## 5round     1     0     3 -0.01944444
\end{verbatim}

Notice how we did not specify the \texttt{attr} argument. As suggested
before, aside from \texttt{dataset} and \texttt{userid} almost all
arguments have a default value and perform a default behavior. When no
input is entered \texttt{attr} calculates using all recognized
attributes and \texttt{rounds} with the first and the last, which is why
we explicitly specified \texttt{"all"}. Our next step is to determine
the reference points. For the \texttt{refps} of PT we will use the
default settings of user \texttt{33} which are:

\begin{Shaded}
\begin{Highlighting}[]
\KeywordTok{decisionMatrix}\NormalTok{(camera_data, }\DecValTok{33}\NormalTok{, }\DataTypeTok{rounds=}\StringTok{"first"}\NormalTok{, }\DataTypeTok{cost_ids=}\DecValTok{4}\NormalTok{)}
\end{Highlighting}
\end{Shaded}

\begin{verbatim}
## $`33`
##        attr1 attr2 attr3      attr4
## 0round     1     1     1 -0.1680556
\end{verbatim}

This result should correspond to and validate our PT-reference-point
function \texttt{referencePoints}.

\begin{Shaded}
\begin{Highlighting}[]
\KeywordTok{referencePoints}\NormalTok{(camera_data, }\DecValTok{33}\NormalTok{, }\DataTypeTok{cost_ids=}\DecValTok{4}\NormalTok{)}
\end{Highlighting}
\end{Shaded}

\begin{verbatim}
## $`33`
##       rp 1       rp 2       rp 3       rp 4 
##  1.0000000  1.0000000  1.0000000 -0.1680556
\end{verbatim}

Now that we have determined the decision matrix and the reference points
for user 33, we can proceed to compute the following steps.

{[}Insert quick figure{]}

However, since we have demonstrated how the functions build on each
other and to avoid repetitiveness, we will calculate these matrices
using only one function.

\begin{Shaded}
\begin{Highlighting}[]
\KeywordTok{pvMatrix}\NormalTok{(camera_data, }\DecValTok{33}\NormalTok{, }\DataTypeTok{attr=}\DecValTok{1}\NormalTok{:}\DecValTok{4}\NormalTok{, }\DataTypeTok{rounds=}\StringTok{"all"}\NormalTok{, }\DataTypeTok{cost_ids =} \DecValTok{4}\NormalTok{,}
         \DataTypeTok{alpha =} \FloatTok{0.88}\NormalTok{, }\DataTypeTok{beta=}\FloatTok{0.88}\NormalTok{, }\DataTypeTok{lambda=}\FloatTok{2.25}\NormalTok{)}
\end{Highlighting}
\end{Shaded}

\begin{verbatim}
## $`33`
##      [,1]  [,2]      [,3]         [,4]
## [1,]    0  0.00 0.0000000 0.000000e+00
## [2,]    0  0.00 0.5433674 1.000000e+00
## [3,]    0 -2.25 0.5433674 2.129512e-12
## [4,]    1 -2.25 0.5433674 1.000000e+00
## [5,]    0 -2.25 0.5433674 2.129512e-12
## [6,]    0 -2.25 1.0000000 1.000000e+00
\end{verbatim}

Later for the sake of consistency, to compare to DRP and TRP do input a
\texttt{refps} for PT.

\end{document}
