\documentclass[]{article}
\usepackage{lmodern}
\usepackage{amssymb,amsmath}
\usepackage{ifxetex,ifluatex}
\usepackage{fixltx2e} % provides \textsubscript
\ifnum 0\ifxetex 1\fi\ifluatex 1\fi=0 % if pdftex
  \usepackage[T1]{fontenc}
  \usepackage[utf8]{inputenc}
\else % if luatex or xelatex
  \ifxetex
    \usepackage{mathspec}
    \usepackage{xltxtra,xunicode}
  \else
    \usepackage{fontspec}
  \fi
  \defaultfontfeatures{Mapping=tex-text,Scale=MatchLowercase}
  \newcommand{\euro}{€}
\fi
% use upquote if available, for straight quotes in verbatim environments
\IfFileExists{upquote.sty}{\usepackage{upquote}}{}
% use microtype if available
\IfFileExists{microtype.sty}{%
\usepackage{microtype}
\UseMicrotypeSet[protrusion]{basicmath} % disable protrusion for tt fonts
}{}
\usepackage[margin=1in]{geometry}
\usepackage{color}
\usepackage{fancyvrb}
\newcommand{\VerbBar}{|}
\newcommand{\VERB}{\Verb[commandchars=\\\{\}]}
\DefineVerbatimEnvironment{Highlighting}{Verbatim}{commandchars=\\\{\}}
% Add ',fontsize=\small' for more characters per line
\usepackage{framed}
\definecolor{shadecolor}{RGB}{248,248,248}
\newenvironment{Shaded}{\begin{snugshade}}{\end{snugshade}}
\newcommand{\KeywordTok}[1]{\textcolor[rgb]{0.13,0.29,0.53}{\textbf{{#1}}}}
\newcommand{\DataTypeTok}[1]{\textcolor[rgb]{0.13,0.29,0.53}{{#1}}}
\newcommand{\DecValTok}[1]{\textcolor[rgb]{0.00,0.00,0.81}{{#1}}}
\newcommand{\BaseNTok}[1]{\textcolor[rgb]{0.00,0.00,0.81}{{#1}}}
\newcommand{\FloatTok}[1]{\textcolor[rgb]{0.00,0.00,0.81}{{#1}}}
\newcommand{\CharTok}[1]{\textcolor[rgb]{0.31,0.60,0.02}{{#1}}}
\newcommand{\StringTok}[1]{\textcolor[rgb]{0.31,0.60,0.02}{{#1}}}
\newcommand{\CommentTok}[1]{\textcolor[rgb]{0.56,0.35,0.01}{\textit{{#1}}}}
\newcommand{\OtherTok}[1]{\textcolor[rgb]{0.56,0.35,0.01}{{#1}}}
\newcommand{\AlertTok}[1]{\textcolor[rgb]{0.94,0.16,0.16}{{#1}}}
\newcommand{\FunctionTok}[1]{\textcolor[rgb]{0.00,0.00,0.00}{{#1}}}
\newcommand{\RegionMarkerTok}[1]{{#1}}
\newcommand{\ErrorTok}[1]{\textbf{{#1}}}
\newcommand{\NormalTok}[1]{{#1}}
\ifxetex
  \usepackage[setpagesize=false, % page size defined by xetex
              unicode=false, % unicode breaks when used with xetex
              xetex]{hyperref}
\else
  \usepackage[unicode=true]{hyperref}
\fi
\hypersetup{breaklinks=true,
            bookmarks=true,
            pdfauthor={Diego Aviles},
            pdftitle={test},
            colorlinks=true,
            citecolor=blue,
            urlcolor=blue,
            linkcolor=magenta,
            pdfborder={0 0 0}}
\urlstyle{same}  % don't use monospace font for urls
\setlength{\parindent}{0pt}
\setlength{\parskip}{6pt plus 2pt minus 1pt}
\setlength{\emergencystretch}{3em}  % prevent overfull lines
\setcounter{secnumdepth}{0}

%%% Use protect on footnotes to avoid problems with footnotes in titles
\let\rmarkdownfootnote\footnote%
\def\footnote{\protect\rmarkdownfootnote}

%%% Change title format to be more compact
\usepackage{titling}

% Create subtitle command for use in maketitle
\newcommand{\subtitle}[1]{
  \posttitle{
    \begin{center}\large#1\end{center}
    }
}

\setlength{\droptitle}{-2em}
  \title{test}
  \pretitle{\vspace{\droptitle}\centering\huge}
  \posttitle{\par}
  \author{Diego Aviles}
  \preauthor{\centering\large\emph}
  \postauthor{\par}
  \predate{\centering\large\emph}
  \postdate{\par}
  \date{Wednesday, July 29, 2015}



\begin{document}

\maketitle


This is an R Markdown document. Markdown is a simple formatting syntax
for authoring HTML, PDF, and MS Word documents. For more details on
using R Markdown see \url{http://rmarkdown.rstudio.com}.

When you click the \textbf{Knit} button a document will be generated
that includes both content as well as the output of any embedded R code
chunks within the document. You can embed an R code chunk like this:

\section{Heather?}\label{heather}

\subsection{Heather?!?!?!?}\label{heather-1}

\begin{Shaded}
\begin{Highlighting}[]
\KeywordTok{install_github}\NormalTok{(}\StringTok{"avilesd/productConfig"}\NormalTok{)}
\KeywordTok{library}\NormalTok{(productConfig)}
\end{Highlighting}
\end{Shaded}

\begin{center}\rule{0.5\linewidth}{\linethickness}\end{center}

\begin{Shaded}
\begin{Highlighting}[]
\KeywordTok{head}\NormalTok{(camera_data)}
\end{Highlighting}
\end{Shaded}

\begin{verbatim}
##   cid usid round atid selected selectable
## 1  13    6     0    1        3          3
## 2  14    6     1    1        3          3
## 3  15    6     2    1        3         -1
## 4  16    6     3    1        3         -1
## 5  17    6     4    1        0          3
## 6  18    6     5    1        0          3
\end{verbatim}

Before calculating the decision matrix, it is necessary to first gather
some key data from the \texttt{camera\_data} provided. This is were the
\texttt{GetFunctions} come in. For example, it is necessary to know how
many attributes there are,

\begin{Shaded}
\begin{Highlighting}[]
\KeywordTok{get_attrs_ID}\NormalTok{(camera_data)}
\end{Highlighting}
\end{Shaded}

\begin{verbatim}
## [1] 1 2 3 4
\end{verbatim}

as well as how the possible value each attribute can have:

\begin{Shaded}
\begin{Highlighting}[]
\KeywordTok{get_attr_values}\NormalTok{(}\DataTypeTok{dataset =} \NormalTok{camera_data, }\DataTypeTok{attrid =} \DecValTok{1}\NormalTok{)}
\KeywordTok{get_attr_values}\NormalTok{(camera_data, }\DecValTok{2}\NormalTok{)}
\KeywordTok{get_attr_values}\NormalTok{(camera_data, }\DecValTok{3}\NormalTok{)}
\KeywordTok{get_attr_values}\NormalTok{(camera_data, }\DecValTok{4}\NormalTok{) ## Price attribute}
\end{Highlighting}
\end{Shaded}

\begin{verbatim}
## [1] 3 0 2 1
\end{verbatim}

\begin{verbatim}
## [1] 0 3 2 1
\end{verbatim}

\begin{verbatim}
## [1] 0 3 2 1
\end{verbatim}

\begin{verbatim}
## [1]  0.16805556 -0.27777778 -0.12916667  0.01944444  0.46527778  0.31666667
## [7]  0.61388889
\end{verbatim}

But even more important, is to know how many \texttt{rounds}, i.e.~how
many rows our decision matrix can have. For the remainder of this
example we are going to work with a random picked \texttt{userid = 18}.

\begin{Shaded}
\begin{Highlighting}[]
\KeywordTok{get_rounds_by_ID}\NormalTok{(camera_data, }\DataTypeTok{userid =} \DecValTok{18}\NormalTok{)}
\end{Highlighting}
\end{Shaded}

\begin{verbatim}
##  [1]  0  1  2  3  4  5  6  7  8  9 10 11 12 13 14
\end{verbatim}

The user 18, interacted 15 times with the camera configurator. This
means, that there were 15 configurations the user considerated before
taking his decision. Note that given the nature of product
configurators, it is likely to see duplicate alternatives, i.e.~equal
configurations.

We know now the number of columns (attributes) and the number of rows
(rounds) the decision matrix can have, for our selected user. For the
decision matrix the user has the ability to choose how many columns and
rounds he wants to use, this is done through the \texttt{attr} and
\texttt{rounds} parameters, respectively.

\end{document}
