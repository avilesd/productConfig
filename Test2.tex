\documentclass[]{article}
\usepackage{lmodern}
\usepackage{amssymb,amsmath}
\usepackage{ifxetex,ifluatex}
\usepackage{fixltx2e} % provides \textsubscript
\ifnum 0\ifxetex 1\fi\ifluatex 1\fi=0 % if pdftex
  \usepackage[T1]{fontenc}
  \usepackage[utf8]{inputenc}
\else % if luatex or xelatex
  \ifxetex
    \usepackage{mathspec}
    \usepackage{xltxtra,xunicode}
  \else
    \usepackage{fontspec}
  \fi
  \defaultfontfeatures{Mapping=tex-text,Scale=MatchLowercase}
  \newcommand{\euro}{€}
\fi
% use upquote if available, for straight quotes in verbatim environments
\IfFileExists{upquote.sty}{\usepackage{upquote}}{}
% use microtype if available
\IfFileExists{microtype.sty}{%
\usepackage{microtype}
\UseMicrotypeSet[protrusion]{basicmath} % disable protrusion for tt fonts
}{}
\usepackage[margin=1in]{geometry}
\usepackage{color}
\usepackage{fancyvrb}
\newcommand{\VerbBar}{|}
\newcommand{\VERB}{\Verb[commandchars=\\\{\}]}
\DefineVerbatimEnvironment{Highlighting}{Verbatim}{commandchars=\\\{\}}
% Add ',fontsize=\small' for more characters per line
\usepackage{framed}
\definecolor{shadecolor}{RGB}{248,248,248}
\newenvironment{Shaded}{\begin{snugshade}}{\end{snugshade}}
\newcommand{\KeywordTok}[1]{\textcolor[rgb]{0.13,0.29,0.53}{\textbf{{#1}}}}
\newcommand{\DataTypeTok}[1]{\textcolor[rgb]{0.13,0.29,0.53}{{#1}}}
\newcommand{\DecValTok}[1]{\textcolor[rgb]{0.00,0.00,0.81}{{#1}}}
\newcommand{\BaseNTok}[1]{\textcolor[rgb]{0.00,0.00,0.81}{{#1}}}
\newcommand{\FloatTok}[1]{\textcolor[rgb]{0.00,0.00,0.81}{{#1}}}
\newcommand{\CharTok}[1]{\textcolor[rgb]{0.31,0.60,0.02}{{#1}}}
\newcommand{\StringTok}[1]{\textcolor[rgb]{0.31,0.60,0.02}{{#1}}}
\newcommand{\CommentTok}[1]{\textcolor[rgb]{0.56,0.35,0.01}{\textit{{#1}}}}
\newcommand{\OtherTok}[1]{\textcolor[rgb]{0.56,0.35,0.01}{{#1}}}
\newcommand{\AlertTok}[1]{\textcolor[rgb]{0.94,0.16,0.16}{{#1}}}
\newcommand{\FunctionTok}[1]{\textcolor[rgb]{0.00,0.00,0.00}{{#1}}}
\newcommand{\RegionMarkerTok}[1]{{#1}}
\newcommand{\ErrorTok}[1]{\textbf{{#1}}}
\newcommand{\NormalTok}[1]{{#1}}
\ifxetex
  \usepackage[setpagesize=false, % page size defined by xetex
              unicode=false, % unicode breaks when used with xetex
              xetex]{hyperref}
\else
  \usepackage[unicode=true]{hyperref}
\fi
\hypersetup{breaklinks=true,
            bookmarks=true,
            pdfauthor={Diego Aviles},
            pdftitle={test},
            colorlinks=true,
            citecolor=blue,
            urlcolor=blue,
            linkcolor=magenta,
            pdfborder={0 0 0}}
\urlstyle{same}  % don't use monospace font for urls
\setlength{\parindent}{0pt}
\setlength{\parskip}{6pt plus 2pt minus 1pt}
\setlength{\emergencystretch}{3em}  % prevent overfull lines
\setcounter{secnumdepth}{0}

%%% Use protect on footnotes to avoid problems with footnotes in titles
\let\rmarkdownfootnote\footnote%
\def\footnote{\protect\rmarkdownfootnote}

%%% Change title format to be more compact
\usepackage{titling}

% Create subtitle command for use in maketitle
\newcommand{\subtitle}[1]{
  \posttitle{
    \begin{center}\large#1\end{center}
    }
}

\setlength{\droptitle}{-2em}
  \title{test}
  \pretitle{\vspace{\droptitle}\centering\huge}
  \posttitle{\par}
  \author{Diego Aviles}
  \preauthor{\centering\large\emph}
  \postauthor{\par}
  \predate{\centering\large\emph}
  \postdate{\par}
  \date{Wednesday, July 29, 2015}



\begin{document}

\maketitle


This is an R Markdown document. Markdown is a simple formatting syntax
for authoring HTML, PDF, and MS Word documents. For more details on
using R Markdown see \url{http://rmarkdown.rstudio.com}.

When you click the \textbf{Knit} button a document will be generated
that includes both content as well as the output of any embedded R code
chunks within the document. You can embed an R code chunk like this:

\section{Heather?}\label{heather}

\subsection{Heather?!?!?!?}\label{heather-1}

\begin{Shaded}
\begin{Highlighting}[]
\KeywordTok{install_github}\NormalTok{(}\StringTok{"avilesd/productConfig"}\NormalTok{)}
\KeywordTok{library}\NormalTok{(productConfig)}
\end{Highlighting}
\end{Shaded}

\begin{center}\rule{0.5\linewidth}{\linethickness}\end{center}

\begin{Shaded}
\begin{Highlighting}[]
\KeywordTok{head}\NormalTok{(camera_data)}
\end{Highlighting}
\end{Shaded}

\begin{verbatim}
##   cid usid round atid selected selectable
## 1  13    6     0    1        3          3
## 2  14    6     1    1        3          3
## 3  15    6     2    1        3         -1
## 4  16    6     3    1        3         -1
## 5  17    6     4    1        0          3
## 6  18    6     5    1        0          3
\end{verbatim}

Before calculating the decision matrix, it is necessary to first gather
some key data from the \texttt{camera\_data} provided. This is were the
\texttt{GetFunctions} come in. For example, it is necessary to know how
many attributes there are,

\begin{Shaded}
\begin{Highlighting}[]
\KeywordTok{get_attrs_ID}\NormalTok{(camera_data)}
\end{Highlighting}
\end{Shaded}

\begin{verbatim}
## [1] 1 2 3 4
\end{verbatim}

as well as how the possible value each attribute can have:

\begin{Shaded}
\begin{Highlighting}[]
\KeywordTok{get_attr_values}\NormalTok{(}\DataTypeTok{dataset =} \NormalTok{camera_data, }\DataTypeTok{attrid =} \DecValTok{1}\NormalTok{)}
\KeywordTok{get_attr_values}\NormalTok{(camera_data, }\DecValTok{2}\NormalTok{)}
\KeywordTok{get_attr_values}\NormalTok{(camera_data, }\DecValTok{3}\NormalTok{)}
\KeywordTok{get_attr_values}\NormalTok{(camera_data, }\DecValTok{4}\NormalTok{) ## Price attribute}
\end{Highlighting}
\end{Shaded}

\begin{verbatim}
## [1] 3 0 2 1
\end{verbatim}

\begin{verbatim}
## [1] 0 3 2 1
\end{verbatim}

\begin{verbatim}
## [1] 0 3 2 1
\end{verbatim}

\begin{verbatim}
## [1]  0.16805556 -0.27777778 -0.12916667  0.01944444  0.46527778  0.31666667
## [7]  0.61388889
\end{verbatim}

But even more important, is to know how many \texttt{rounds}, i.e.~how
many rows our decision matrix can have. For the remainder of this
example we are going to work with a random picked \texttt{userid = 18}.

\begin{Shaded}
\begin{Highlighting}[]
\KeywordTok{get_rounds_by_ID}\NormalTok{(camera_data, }\DataTypeTok{userid =} \DecValTok{18}\NormalTok{)}
\end{Highlighting}
\end{Shaded}

\begin{verbatim}
##  [1]  0  1  2  3  4  5  6  7  8  9 10 11 12 13 14
\end{verbatim}

The user 18, interacted 15 times with the camera configurator. This
means, that there were 15 configurations the user considerated before
taking his decision. Note that given the nature of product
configurators, it is likely to see duplicate alternatives, i.e.~equal
configurations.

We know now the number of columns (attributes) and the number of rows
(rounds) the decision matrix can have, for our selected user. For the
decision matrix the reader has the ability to choose how many columns
and rounds he wants to use, this is done through the \texttt{attr} and
\texttt{rounds} parameters, respectively. To calculate the decision
matrix using all attributes and rounds, we can find at least three
equivalent ways to do it:

\begin{Shaded}
\begin{Highlighting}[]
\KeywordTok{decision_matrix}\NormalTok{(}\DataTypeTok{data=} \NormalTok{camera_data, }\DataTypeTok{userid=} \DecValTok{18}\NormalTok{,  }\DataTypeTok{attr=} \KeywordTok{c}\NormalTok{(}\DecValTok{1}\NormalTok{,}\DecValTok{2}\NormalTok{,}\DecValTok{3}\NormalTok{,}\DecValTok{4}\NormalTok{), }
                \DataTypeTok{rounds=} \KeywordTok{c}\NormalTok{(}\DecValTok{0}\NormalTok{, }\DecValTok{1}\NormalTok{, }\DecValTok{2}\NormalTok{, }\DecValTok{3}\NormalTok{, }\DecValTok{4}\NormalTok{, }\DecValTok{5}\NormalTok{, }\DecValTok{6}\NormalTok{, }\DecValTok{7}\NormalTok{, }\DecValTok{8}\NormalTok{, }\DecValTok{9}\NormalTok{, }\DecValTok{10}\NormalTok{, }\DecValTok{11}\NormalTok{, }\DecValTok{12}\NormalTok{, }\DecValTok{13}\NormalTok{, }\DecValTok{14}\NormalTok{))}

\KeywordTok{decision_matrix}\NormalTok{(}\DataTypeTok{data=} \NormalTok{camera_data, }\DecValTok{18}\NormalTok{, }\DataTypeTok{attr=} \KeywordTok{get_attrs_ID}\NormalTok{(camera_data),}
                \DataTypeTok{rounds =} \KeywordTok{get_rounds_by_ID}\NormalTok{(camera_data, }\DataTypeTok{userid=} \DecValTok{18}\NormalTok{))}

\KeywordTok{decision_matrix}\NormalTok{(}\DataTypeTok{data=} \NormalTok{camera_data, }\DecValTok{18}\NormalTok{, }\DataTypeTok{rounds=}\StringTok{"all"}\NormalTok{)}
\end{Highlighting}
\end{Shaded}

Returning the full decision matrix for the selected user:

\begin{verbatim}
##         attr1 attr2 attr3      attr4
## round0      2     0     0 0.31666667
## round1      2     1     0 0.16805556
## round2      2     0     0 0.31666667
## round3      2     0     2 0.01944444
## round4      0     0     2 0.31666667
## round5      0     0     3 0.16805556
## round6      0     0     2 0.31666667
## round7      0     0     1 0.46527778
## round8      0     0     2 0.31666667
## round9      0     2     2 0.01944444
## round10     0     0     2 0.31666667
## round11     1     0     2 0.16805556
## round12     0     0     2 0.31666667
## round13     0     0     3 0.16805556
## round14     0     0     2 0.31666667
\end{verbatim}

Note that in the third option, we did not enter any \texttt{attr} value.
Excluding the \texttt{dataset} and \texttt{userid} arguments, all
arguments are assigned a default value by the functions, unless stated
otherwise by the reader. In this case, the default behavior without
\texttt{attr} input is to calculate all attributes found in the given
data. At this point it is important to note that default values and
behavior can be changed, but should be well documented.

As the next step in our procedure comes the calculation of the reference
point vector, \texttt{refps} in `productConfig'. Ideally, the length of
the \texttt{refps} vector is equal to the length of the \texttt{attr}
vector, although the function \texttt{ref\_points} has built-in
capabilites to handle other scenarios. The package's user may enter its
own values. If no value is given, \texttt{refps} determines the
reference points as the default values for each attribute in the initial
configuration. For our selected user they are:

\begin{Shaded}
\begin{Highlighting}[]
\KeywordTok{ref_points}\NormalTok{(camera_data, }\DataTypeTok{userid=} \DecValTok{18}\NormalTok{) }
\end{Highlighting}
\end{Shaded}

\begin{verbatim}
##     rp 1      rp 2      rp 3      rp 4  
## 2.0000000 0.0000000 0.0000000 0.3166667
\end{verbatim}

We discussed earlier the possibility of reference points being defined
by the decision-maker as other than default values. There is also
extensive literature about the use of multiple reference points, for a
brief overview see {[}11{]}. At this point of development we worked only
with single reference points ,default values, or status quo. It is but a
major limitation and it presents great room for future package
development in this direction.

Having determined the decision matrix and the reference points vector,
we now calculate the normalized gain and loss matrices. After the next
code sample we leave the attr and refps command out, since their default
values calculate the same values we are using.

\begin{Shaded}
\begin{Highlighting}[]
\KeywordTok{norm_g_l_matrices}\NormalTok{(}\DataTypeTok{data=} \NormalTok{camera_data, }\DataTypeTok{userid=} \DecValTok{18}\NormalTok{,}
                              \DataTypeTok{attr =} \KeywordTok{get_attrs_ID}\NormalTok{(camera_data),  }
                              \DataTypeTok{rounds =} \StringTok{"all"}\NormalTok{,  }
                              \DataTypeTok{refps =} \OtherTok{NULL}\NormalTok{) }
\end{Highlighting}
\end{Shaded}

\begin{verbatim}
## $ngain
##       [,1] [,2]      [,3] [,4]
##  [1,]    0  0.0 0.0000000  0.0
##  [2,]    0  0.5 0.0000000  0.0
##  [3,]    0  0.0 0.0000000  0.0
##  [4,]    0  0.0 0.6666667  0.0
##  [5,]    0  0.0 0.6666667  0.0
##  [6,]    0  0.0 1.0000000  0.0
##  [7,]    0  0.0 0.6666667  0.0
##  [8,]    0  0.0 0.3333333  0.5
##  [9,]    0  0.0 0.6666667  0.0
## [10,]    0  1.0 0.6666667  0.0
## [11,]    0  0.0 0.6666667  0.0
## [12,]    0  0.0 0.6666667  0.0
## [13,]    0  0.0 0.6666667  0.0
## [14,]    0  0.0 1.0000000  0.0
## [15,]    0  0.0 0.6666667  0.0
## 
## $nloss
##       [,1] [,2] [,3]          [,4]
##  [1,]  0.0    0    0  0.000000e+00
##  [2,]  0.0    0    0 -5.000000e-01
##  [3,]  0.0    0    0  0.000000e+00
##  [4,]  0.0    0    0 -1.000000e+00
##  [5,] -1.0    0    0 -3.810036e-14
##  [6,] -1.0    0    0 -5.000000e-01
##  [7,] -1.0    0    0 -3.810036e-14
##  [8,] -1.0    0    0  0.000000e+00
##  [9,] -1.0    0    0 -3.810036e-14
## [10,] -1.0    0    0 -1.000000e+00
## [11,] -1.0    0    0 -3.810036e-14
## [12,] -0.5    0    0 -5.000000e-01
## [13,] -1.0    0    0 -3.810036e-14
## [14,] -1.0    0    0 -5.000000e-01
## [15,] -1.0    0    0 -3.810036e-14
\end{verbatim}

For the final step of calculating the prospect values, we are only
missing the attribute weights. The \texttt{WeightFunctions} provides
different functions to calculate the weights. It is structured in such a
way that it does this with what we call an interface function, named
\texttt{get\_attr\_weight}. The idea is to simplify the calculation by
providing the interface function with the necessary parameters and the
name of the function you want to use. Nevertheless at the time of the
publication of this seminar paper, only one weight function was deemed
mature enough to include.

Weight functions are also a point to reflect on. To the best of my
knowledge, there is no uniformly consensus on how to determine them. We
are working in some creative ways to calculate them using the dataset
the reader provides. However, we encourage more thoughts on this
subject, to advance the status of this package. The standard function
used at the moment \texttt{weight\_higher\_sum\_value} weights each
attribute according to the relative size of the sum of its values across
all rounds.

\begin{Shaded}
\begin{Highlighting}[]
\NormalTok{weights_demo <-}\StringTok{ }\KeywordTok{get_attr_weight}\NormalTok{(}\DataTypeTok{dataset=} \NormalTok{camera_data, }\DataTypeTok{userid=} \DecValTok{18}\NormalTok{,}
                \DataTypeTok{weight=} \OtherTok{NULL}\NormalTok{,}
                \DataTypeTok{rounds=} \StringTok{"all"}\NormalTok{)}
\NormalTok{weights_demo}
\end{Highlighting}
\end{Shaded}

\begin{verbatim}
##      attr1      attr2      attr3      attr4 
## 0.11841074 0.03947025 0.32891873 0.51320027
\end{verbatim}

As you can observe, all weights are smaller than 1 and their sum
\texttt{sum(weights\_demo)} returns 1.

We have gone through all the necessary functions that lead to the higher
level cluster \texttt{ProspectValueFunctions}. Within this group of
functions we find two important steps, (1)the calculation of the value
matrix and the following (2) determination of the overall prospect
values for each one of the fifteen rounds.

\textbf{Value matrix}

\begin{Shaded}
\begin{Highlighting}[]
\KeywordTok{pvalue_matrix}\NormalTok{(}\DataTypeTok{dataset=} \NormalTok{camera_data, }\DataTypeTok{userid=} \DecValTok{18}\NormalTok{,}
                \DataTypeTok{rounds=} \StringTok{"all"}\NormalTok{,}
                \DataTypeTok{alpha=} \FloatTok{0.88}\NormalTok{, }\DataTypeTok{beta=} \FloatTok{0.88}\NormalTok{, }\DataTypeTok{lambda=} \FloatTok{2.25}\NormalTok{)  ## attr, refps default}
\end{Highlighting}
\end{Shaded}

\begin{verbatim}
##            [,1]      [,2]      [,3]          [,4]
##  [1,]  0.000000 0.0000000 0.0000000  0.000000e+00
##  [2,]  0.000000 0.5433674 0.0000000 -1.222577e+00
##  [3,]  0.000000 0.0000000 0.0000000  0.000000e+00
##  [4,]  0.000000 0.0000000 0.6999060 -2.250000e+00
##  [5,] -2.250000 0.0000000 0.6999060 -3.494622e-12
##  [6,] -2.250000 0.0000000 1.0000000 -1.222577e+00
##  [7,] -2.250000 0.0000000 0.6999060 -3.494622e-12
##  [8,] -2.250000 0.0000000 0.3803061  5.433674e-01
##  [9,] -2.250000 0.0000000 0.6999060 -3.494622e-12
## [10,] -2.250000 1.0000000 0.6999060 -2.250000e+00
## [11,] -2.250000 0.0000000 0.6999060 -3.494622e-12
## [12,] -1.222577 0.0000000 0.6999060 -1.222577e+00
## [13,] -2.250000 0.0000000 0.6999060 -3.494622e-12
## [14,] -2.250000 0.0000000 1.0000000 -1.222577e+00
## [15,] -2.250000 0.0000000 0.6999060 -3.494622e-12
\end{verbatim}

\textbf{Overall prospect values}

\begin{Shaded}
\begin{Highlighting}[]
\KeywordTok{overall_pv}\NormalTok{(}\DataTypeTok{dataset=} \NormalTok{camera_data, }\DataTypeTok{userid=} \DecValTok{18}\NormalTok{,}
                \DataTypeTok{rounds =} \StringTok{"all"}\NormalTok{,}
                \DataTypeTok{alpha =} \FloatTok{0.88}\NormalTok{, }\DataTypeTok{beta =} \FloatTok{0.88}\NormalTok{, }\DataTypeTok{lambda =} \FloatTok{2.25}\NormalTok{)  ## attr, refps, weight default}
\end{Highlighting}
\end{Shaded}

\begin{verbatim}
##  [1]  0.00000000 -0.60597986  0.00000000 -0.92448843 -0.03621199
##  [6] -0.56493215 -0.03621199  0.13752194 -0.03621199 -1.15144236
## [11] -0.03621199 -0.54198074 -0.03621199 -0.56493215 -0.03621199
\end{verbatim}

Although this walk-through has allowed us to ilustrate some
functionality of our package, we capitalize this moment to show off the
reasoning behind the strutcuring of `productConfig'. Since the higher
level function \texttt{overall\_pv} is independent from the steps shown
above, we can reproduce the exact same result just by running a simple
command.

\begin{Shaded}
\begin{Highlighting}[]
\KeywordTok{overall_pv}\NormalTok{(camera_data, }\DecValTok{18}\NormalTok{, }\DataTypeTok{rounds =} \StringTok{"all"}\NormalTok{) }
\end{Highlighting}
\end{Shaded}

\begin{verbatim}
##  [1]  0.00000000 -0.60597986  0.00000000 -0.92448843 -0.03621199
##  [6] -0.56493215 -0.03621199  0.13752194 -0.03621199 -1.15144236
## [11] -0.03621199 -0.54198074 -0.03621199 -0.56493215 -0.03621199
\end{verbatim}

One problem the reader might suggest is that the results are only for
one user. This is where the \texttt{powerful\_function} is particularly
useful.

\begin{Shaded}
\begin{Highlighting}[]
\KeywordTok{powerful_function}\NormalTok{(camera_data,}
                  \DataTypeTok{userid=} \KeywordTok{get_all_userids}\NormalTok{(camera_data), }
                  \DataTypeTok{FUN=} \NormalTok{overall_pv,}
                  \DataTypeTok{rounds =} \StringTok{"all"}\NormalTok{)}
\end{Highlighting}
\end{Shaded}

For space purposes we will only show the results of four randomly picked
users.

\begin{Shaded}
\begin{Highlighting}[]
\KeywordTok{powerful_function}\NormalTok{(camera_data,}
                  \DataTypeTok{userid=} \KeywordTok{c}\NormalTok{(}\DecValTok{54}\NormalTok{, }\DecValTok{20}\NormalTok{, }\DecValTok{6}\NormalTok{, }\DecValTok{16}\NormalTok{), }
                  \DataTypeTok{FUN=} \NormalTok{overall_pv,}
                  \DataTypeTok{rounds =} \StringTok{"all"}\NormalTok{)}
\end{Highlighting}
\end{Shaded}

\begin{verbatim}
## $usid54
## [1]  0.0000000 -0.2930264  0.0000000 -0.2930264  0.0000000 -0.2930264
## [7]  0.0000000
## 
## $usid20
##  [1]  0.0000000000 -0.0024465102 -0.3477875648 -0.0024465102 -0.3711682414
##  [6] -0.0024465102 -0.0009304227 -0.5092419678 -0.0009304227 -0.1411575478
## [11] -0.1117492972 -0.6200608423 -0.2430846082 -0.6200608423 -0.2430846082
## [16] -0.5089079865 -0.2430846082
## 
## $usid6
##   [1]  0.00000000 -0.52240541 -0.36563466 -0.19867397 -0.21957955
##   [6] -0.24646294 -0.29983047 -0.21571471 -0.29983047 -0.21571471
##  [11] -0.43520366 -0.21571471  0.00000000 -0.52240541 -0.36563466
##  [16] -0.19867397 -0.11402736 -0.19867397 -0.36563466 -0.19867397
##  [21] -0.36563466 -0.52240541 -0.36563466 -0.19867397 -0.21957955
##  [26] -0.29983047 -0.21957955 -0.19867397 -0.36563466 -0.19867397
##  [31] -0.16168421 -0.50327560 -0.67023629 -0.50327560 -0.82700704
##  [36] -0.50327560 -0.30460163 -0.56527808 -0.15098001  0.00000000
##  [41] -0.25015722 -0.27106280 -0.25015722 -0.27106280 -0.43520366
##  [46] -0.21571471  0.00000000 -0.21571471  0.00000000 -0.36563466
##  [51] -0.19867397 -0.16168421 -0.20985313 -0.42990489 -0.15098001
##  [56] -0.21571471 -0.29983047 -0.54998768 -0.29983047 -0.42990489
##  [61] -0.20985313 -0.46001035 -0.24515720 -0.50327560 -0.30460163
##  [66] -0.56527808 -0.29983047 -0.54998768 -0.29983047 -0.42990489
##  [71] -0.20985313 -0.15098001 -0.30460163 -0.34121144 -0.58805432
##  [76] -0.34121144 -0.42362901 -0.57441052 -0.42362901 -0.08203762
##  [81] -0.21571471 -0.29983047 -0.24646294 -0.58805432 -0.32993592
##  [86] -0.46001035 -0.32993592 -0.58805432 -0.32993592 -0.46001035
##  [91] -0.20985313 -0.46001035 -0.21316746 -0.46001035 -0.20985313
##  [96] -0.42990489 -0.15098001 -0.21571471 -0.15098001 -0.21571471
## 
## $usid16
## [1]  0.0000000 -0.4119789 -0.4724162 -0.2561427 -0.3307635  0.0000000
## [7] -0.3878132
\end{verbatim}

\end{document}
